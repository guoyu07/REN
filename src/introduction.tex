\part{Introduction}




In computing, a regular expression (abbreviated regex or regexp) is a sequence of characters that forms a search pattern, mainly for use in pattern matching with strings, or string matching, i.e. ``find and replace"-like operations. The concept arose in the 1950s, when the American mathematician Stephen Kleene formalized the description of a regular language, and came into common use with the Unix text processing utilities ed, an editor, and grep (global regular expression print), a filter.

Each character in a regular expression is either understood to be a metacharacter with its special meaning, or a regular character with its literal meaning. Together, they can be used to identify textual material of a given pattern, or process a number of instances of it that can vary from a precise equality to a very general similarity of the pattern. The pattern sequence itself is an expression that is a statement in a language designed specifically to represent prescribed targets in the most concise and flexible way to direct the automation of text processing of general text files, specific textual forms, or of random input strings.

A very simple use of a regular expression would be to locate the same word spelled two different ways in a text editor, for example \textcolor{Blue}{\texttt{seriali[sz]e}}. A wildcard match can also achieve this, but wildcard matches differ from regular expressions in that wildcards are limited to what they can pattern (having fewer metacharacters and a simple language-base), whereas regular expressions are not. A usual context of wildcard characters is in globbing similar names in a list of files, whereas regular expressions are usually employed in applications that pattern-match text strings in general. For example, the simple regexp \verb|^[ \t]+||\verb|[ \t]+$| matches excess whitespace at the beginning and end of a line. An advanced regexp used to match any numeral is \verb|^[+-]?(\d+\.?\d*|\verb|\.\d+)([eE][+-]?\d+)?$|.


A regular expression processor processes a regular expression statement expressed in terms of a grammar in a given formal language, and with that examines the target text string, parsing it to identify substrings that are members of its language, the regular expressions.

Regular expressions are so useful in computing that the various systems to specify regular expressions have evolved to provide both a basic and extended standard for the grammar and syntax; modern regular expressions heavily augment the standard. Regular expression processors are found in several search engines, search and replace dialogs of several word processors and text editors, and in the command lines of text processing utilities, such as sed and AWK.

Many programming languages provide regular expression capabilities, some built-in, for example Perl, Ruby, AWK, and Tcl, and others via a standard library, for example .NET languages, Java, Python and C++ (since C++11). Most other languages offer regular expressions via a library.


\begin{figure}[!ht]
\centering
\includegraphics[scale=0.5]{The_river_effect_in_justified_text.jpg}
\vspace{-20pt}
\caption{\small{A regular expression like \textcolor{Blue}{\texttt{(?<=\textbackslash.) \{2,\}(?=[A-Z])}} would match the highlighted text above.}}
\label{regexp_example}
\end{figure}

注:上述正则表达式的意义如下:

\begin{compactenum}
\item \textcolor{Blue}{\texttt{(?<=\textbackslash.)}} 指定从每句话的结束符开始;
\item $\sqcup$\textcolor{Blue}{\texttt{\{2,\}}}指定只有具有多于两个空格时,才进行操作;
\item \textcolor{Blue}{\texttt{(?=[A-Z])}}指定空格至每一句的开头的大写字母为止。
\end{compactenum}



\chapter{History}


The origins of regular expressions lie in automata theory and formal language theory, both of which are part of theoretical computer science. These fields study models of computation (automata) and ways to describe and classify formal languages. In the 1950s, mathematician Stephen Cole Kleene described these models using his mathematical notation called \textsl{regular sets}. 

The SNOBOL language was an early implementation of pattern matching, but not identical to regular expressions. Among the first appearances of regular expressions in program form was when Ken Thompson built Kleene's notation into the editor QED as a means to match patterns in text files. He later added this capability to the Unix editor ed, which eventually led to the popular search tool grep's use of regular expressions (``grep" is a word derived from the command for regular expression searching in the ed editor: \texttt{g/re/p} meaning ``Global search for Regular Expression and Print matching lines"). Around the same time when Thompson developed QED, a group of researchers including Douglas T. Ross implemented a tool based on regular expressions that is used for lexical analysis in compiler design.[5] Since that time, many variations of the original adaptations of regular expressions have been widely used in Unix and Unix-like utilities including expr, AWK, Emacs, vi, and lex.

Perl regular expressions derive from a regex library written by Henry Spencer, who later wrote an implementation of Advanced Regular Expressions for Tcl. The Tcl library is a hybrid NFA/DFA implementation with improved performance characteristics, earning praise from Jeffrey Friedl who said, ``...it really seems quite wonderful." Software projects that have adopted Spencer's Tcl regular expression implementation include PostgreSQL. Perl later expanded on Spencer's original library to add many new features, but has not yet caught up with Spencer's Advanced Regular Expressions implementation in terms of performance or Unicode handling. Philip Hazel developed PCRE (Perl Compatible Regular Expressions), which attempts to closely mimic Perl's regular expression functionality and is used by many modern tools including PHP and Apache HTTP Server. Part of the effort in the design of Perl 6 is to improve Perl's regular expression integration, and to increase their scope and capabilities to allow the definition of parsing expression grammars. The result is a mini-language called Perl 6 rules, which are used to define Perl 6 grammar as well as provide a tool to programmers in the language. These rules maintain existing features of Perl 5.x regular expressions, but also allow BNF-style definition of a recursive descent parser via sub-rules.

The use of regular expressions in structured information standards for document and database modeling started in the 1960s and expanded in the 1980s when industry standards like ISO SGML (precursored by ANSI ``GCA 101-1983") consolidated. The kernel of the structure specification language standards consists of regular expressions. Its use is evident in the DTD element group syntax.




\chapter{Basic concepts}


A regular expression, often called a \textsf{pattern}, is an expression used to specify a set of strings required for a particular purpose. A simple way to specify a set of strings is simply to list its elements or members. However, there are often more concise ways to specify the desired set of strings. For example, the set containing the three strings ``Handel", ``Händel", and ``Haendel" can be specified by the \textsf{pattern} \textcolor{Blue}{\textsf{H(ä|ae?)ndel;}} we say that this pattern \textsf{matches} each of the three strings. In most formalisms, if there exists at least one regex that matches a particular set then there exists an infinite number of other regex that also match it—the specification is not unique. Most formalisms provide the following operations to construct regular expressions.

\begin{compactitem}
\item[1.~\textsf{Boolean ``or"}]  

A vertical bar(\textsf{|}) separates alternatives. For example, \textsf{pattern} \textcolor{Blue}{\textsf{gray|grey}} can match ``\textsf{gray}" or ``\textsf{grey}".

\item[2.~\textsf{Grouping}]

Parentheses are used to define the scope and precedence of the operators (among other uses). For example, \textsf{pattern} \textcolor{Blue}{\textsf{gray|grey}} and \textsf{pattern} \textcolor{Blue}{\textsf{gr(a|e)y}} are equivalent patterns which both describe the set of ``gray" or ``grey".

\item[3.~\textsf{Grouping}]

A quantifier after a token (such as a character) or group specifies how often that preceding element is allowed to occur. The most common quantifiers are the question mark \textsf{?}, the asterisk \textsf{*} (derived from the Kleene star), and the plus sign \textsf{+} (Kleene cross).

\begin{compactitem}
\item[\textsf{?}] The question mark indicates there is zero or one of the preceding element. For example, \textsf{colou?r} matches both ``\textsf{color}" and ``\textsf{colour}".
\item[\textsf{*}] The asterisk indicates there is zero or more of the preceding element. For example, \textsf{ab*c} matches ``\textsf{ac}", ``\textsf{abc}", ``\textsf{abbc}", ``\textsf{abbbc}", and so on.
\item[\textsf{+}] The plus sign indicates there is one or more of the preceding element. For example, \textsf{ab+c} matches ``\textsf{abc}", ``\textsf{abbc}", ``\textsf{abbbc}", and so on, but not ``\textsf{ac}".
\end{compactitem}

\end{compactitem}


These constructions can be combined to form arbitrarily complex expressions, much like one can construct arithmetical expressions from numbers and the operations +, −, ×, and ÷. For example, \textsf{H(ae?|ä)ndel} and \textsf{H(a|ae|ä)ndel} are both valid patterns which match the same strings as the earlier example, \textsf{H(ä|ae?)ndel}.

The precise syntax for regular expressions varies among tools and with context; more detail is given in the \hyperref[regular_expression_syntax]{Syntax section}.




\chapter{Formal language theory}

Regular expressions describe regular languages in formal language theory. They have the same expressive power as regular grammars.






\section{Formal definition}

Regular expressions consist of constants and operator symbols that denote sets of strings and operations over these sets, respectively. The following definition is standard, and found as such in most textbooks on formal language theory. Given a finite alphabet Σ, the following constants are defined as regular expressions:


\begin{compactitem}
\item (\textsf{empty set}) $\emptyset$ denoting the set $\emptyset$.
\item (\textsf{empty string}) $\varepsilon$ denoting the set containing only the ``empty" string, which has no characters at all.
\item (\textsf{literal character}) a in Σ denoting the set containing only the character a.
\end{compactitem}


Given regular expressions R and S, the following operations over them are defined to produce regular expressions:

\begin{compactitem}
\item (\textsf{concatenation}) RS denoting the set \{ αβ | α in set described by expression R and β in set described by S \}. For example $$\text{\texttt{\{"ab","c"\}\{"d","ef"\} = \{"abd","abef","cd","cef"\}}}$$
\item (\textsf{alternation}) R | S denoting the set union of sets described by R and S. For example, 

if R describes \texttt{\{"ab","c"\}} and S describes \texttt{\{"ab", "d", "ef"\}}, expression R | S describes \texttt{\{"ab", "c", "d", "ef"\}}.$$\text{\texttt{\{"ab","c"\}}|\texttt{\{"ab","d","ef"\}}}=\texttt{\{"ab","c","d","ef"\}}$$
\item (\textsf{Kleene star}) R* denoting the smallest superset of set described by R that contains ε and is closed under string concatenation. This is the set of all strings that can be made by concatenating any finite number (including zero) of strings from set described by R. For example, \texttt{\{"0","1"\}*} is the set of all finite binary strings (including the empty string), and 
$$\text{\texttt{\{"ab","c"\}* = \{}} \varepsilon \text{\texttt{,"ab","c","abab","abc","cab","cc","abaab","abcb",...\}}}$$
\end{compactitem}

To avoid parentheses it is assumed that the Kleene star has the highest priority, then concatenation and then alternation. If there is no ambiguity then parentheses may be omitted. For example, \texttt{(ab)c} can be written as \text{abc}, and \texttt{a|(b(c*))} can be written as \texttt{a|bc*}. Many textbooks use the symbols $\cup$, $+$, or $\vee$ for alternation instead of the vertical bar.

Examples:

\begin{compactitem}
\item \texttt{a|b*} denotes $$\text{\texttt{\{}} \varepsilon \text{\texttt{,"a","b","bb","bbb",...\}}}$$
\item \texttt{(a|b)*} denotes the set of all strings with no symbols other than ``\texttt{a}" and ``\texttt{b}", including the empty string: $$\text{\texttt{\{}} \varepsilon \text{\texttt{,"a","b","aa","ab","ba","bb","aaa",...\}}}$$
\item \texttt{ab*(c|ε)} denotes the set of strings starting with ``\texttt{a}", then zero or more ``\texttt{b}"s and finally optionally a ``\texttt{c}": $$\text{\texttt{\{"a","ac","ab","abc","abb","abbc",...\}}}$$

\end{compactitem}




\section{Expressive power and compactness}

The formal definition of regular expressions is purposely parsimonious and avoids defining the redundant quantifiers \texttt{?} and \texttt{+}, which can be expressed as follows: \texttt{a+ = aa*}, and \texttt{a? = (a|$\varepsilon$)}. Sometimes the complement operator is added, to give a generalized regular expression; here Rc matches all strings over \texttt{Σ*} that do not match R. In principle, the complement operator is redundant, as it can always be circumscribed by using the other operators. However, the process for computing such a representation is complex, and the result may require expressions of a size that is double exponentially larger.

Regular expressions in this sense can express the regular languages, exactly the class of languages accepted by deterministic finite automata. There is, however, a significant difference in compactness. Some classes of regular languages can only be described by deterministic finite automata whose size grows exponentially in the size of the shortest equivalent regular expressions. The standard example here is the languages Lk consisting of all strings over the alphabet \texttt{\{a,b\}} whose kth-from-last letter equals a. On one hand, a regular expression describing L4 is given by $$\text{\texttt{|(a|b)\^{}*a(a|b)(a|b)(a|b)|}}$$


Generalizing this pattern to $L_k$ gives the expression$$(a|b)^*a\underbrace{(a|b)(a|b)\cdots(a|b)}_{k-1\text{  times}}$$

On the other hand, it is known that every deterministic finite automaton accepting the language Lk must have at least 2k states. Luckily, there is a simple mapping from regular expressions to the more general nondeterministic finite automata (NFAs) that does not lead to such a blowup in size; for this reason NFAs are often used as alternative representations of regular languages. NFAs are a simple variation of the type-3 grammars of the Chomsky hierarchy.

Finally, it is worth noting that many real-world ``regular expression" engines implement features that cannot be described by the regular expressions in the sense of formal language theory.




\section{Deciding equivalence of regular expressions}

As seen in many of the examples above, there is more than one way to construct a regular expression to achieve the same results.

It is possible to write an algorithm which for two given regular expressions decides whether the described languages are essentially equal, reduces each expression to a minimal deterministic finite state machine, and determines whether they are isomorphic (equivalent).

The redundancy can be eliminated by using Kleene star and set union to find an interesting subset of regular expressions that is still fully expressive, but perhaps their use can be restricted. This is a surprisingly difficult problem. As simple as the regular expressions are, there is no method to systematically rewrite them to some normal form. The lack of axiom in the past led to the star height problem. In 1991, Dexter Kozen axiomatized regular expressions with Kleene algebra.



\chapter{Syntax}
\label{regular_expression_syntax}


A regexp pattern matches a target string. The pattern is composed of a sequence of atoms. An atom is what matches at a point in the target string. The simplest atom is a literal, but grouping parts of the pattern to match an atom will require using \texttt{( )} as metacharacters. Metacharacters help form: atoms; quantifiers telling how many atoms (and whether it is a greedy quantifier or not); a logical OR character, which offers a set of alternatives, and a logical NOT character, which negates an atom's existence; and back references to refer to previous atoms of a completing pattern of atoms. A match is made, not when all the atoms of the string are matched, but rather when all the pattern atoms in the regular expression have matched. The idea is to make a small pattern of characters stand for a large number of possible strings, rather than compiling a large list of all the literal possibilities.

Depending on the regexp processor there are about fourteen metacharacters, characters that may or may not have their literal character meaning, depending on context, or whether they are ``escaped", i.e., preceded by an escape sequence, in this case, the backslash \textbackslash. Modern and POSIX extended regular expressions use metacharacters more often than their literal meaning, so to avoid ``backslash-osis" it makes sense to have a metacharacter escape to a literal mode; but starting out, it makes more sense to have the four bracketing metacharacters \texttt{( )} and \texttt{\{ \}} be primarily literal, and ``escape" that usual meaning to become metacharacters. Common standards implement both. The usual metacharacters are \texttt{\{\}[]()\^{}\$.|*+?} and \textbackslash. The usual characters that become metacharacters when escaped are dsw.DSW and N.



\section{Delimiters}


When entering a regular expression in a programming language, they may be represented as a usual string literal, hence usually quoted; this is common in C, Java, and Python for instance, where the regular expression re is entered as ``re". However, they are often written with slashes as delimiters, as in \texttt{/re/} for the regular expression re. This originates in ed, where \texttt{/} is the editor command for searching, and an expression \texttt{/re/} can be used to specify a range of lines (matching the pattern), which can be combined with other commands on either side, most famously \texttt{g/re/p} as in grep (``global regex print"). A similar convention is used in sed, where search and replace is given by s/regexp/replacement/ and patterns can be joined with a comma to specify a range of lines as in \texttt{/re1/,/re2/}. This notation is particularly well-known due to its use in Perl, where it forms part of the syntax distinct from normal string literals. In some cases, such as sed and Perl, alternative delimiters can be used to avoid collision with contents, and to avoid having to escape the delimiters. For example, in sed the command \texttt{s,/,X,} will replace a \texttt{/} with an \texttt{X}, using commas as delimiters.


\section{Standards}

The IEEE POSIX standard has three sets of compliance: BRE, ERE, and SRE for Basic, Extended, and Simple Regular Expressions. SRE is deprecated, in favor of BRE, as both provide backward compatibility. The subsection below covering the character classes applies to both BRE and ERE.

BRE and ERE work together. ERE adds \texttt{?, +,} and \texttt{|,} and it removes the need to escape the metacharacters \texttt{( )} and \texttt{\{ \}}, which are required in BRE. Furthermore, as long as the POSIX standard syntax for regular expressions is adhered to, there can be, and often is, additional syntax to serve specific (yet POSIX compliant) applications. Although POSIX.2 leaves some implementation specifics undefined, BRE and ERE provide a ``standard" which has since been adopted as the default syntax of many tools, where the choice of BRE or ERE modes is usually a supported option. For example, GNU grep has the following options: ``\texttt{grep -E}" for ERE, and "``\texttt{grep -G}" for BRE (the default), and ``\texttt{grep -P}" for Perl regular expressions.

Perl regular expressions have become a de facto standard, having a rich and powerful set of atomic expressions. Perl has no ``basic" "extended" level, where the \texttt{( )} and \texttt{\{ \}} may or may not have literal meanings. They are always metacharacters, as they are in ``extended" mode for POSIX. To get their literal meaning, you escape them. Other metacharacters are known to be literal or symbolic based on context alone. Perl offers much more functionality: ``lazy" regular expressions, backtracking, named capture groups, and recursive patterns, all of which are powerful additions to POSIX BRE/ERE.



\subsection{POSIX basic and extended}

In the POSIX standard, Basic Regular Syntax, BRE, requires that the metacharacters \texttt{( )} and \texttt{\{ \}} be designated \texttt{\textbackslash(\textbackslash)} and \texttt{\textbackslash\{\textbackslash\}}, whereas Extended Regular Syntax, ERE, does not.


\subsection{POSIX extended}




\subsection{Character classes}



\section{Standard Perl}




\section{Lazy quantification}







\chapter{Patterns for non-regular languages}






\chapter{Fuzzy regular expressions}







\chapter{Implementations and running times}





\chapter{Unicode}







\chapter{Uses}






\chapter{Examples}




























































\bibliographystyle{plainnat}
\bibliography{REN}
\clearpage